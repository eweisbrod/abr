% Options for packages loaded elsewhere
\PassOptionsToPackage{unicode}{hyperref}
\PassOptionsToPackage{hyphens}{url}
\PassOptionsToPackage{dvipsnames,svgnames,x11names}{xcolor}
%
\documentclass[
  authoryear,
  preprint]{elsarticle}

\usepackage{amsmath,amssymb}
\usepackage{iftex}
\ifPDFTeX
  \usepackage[T1]{fontenc}
  \usepackage[utf8]{inputenc}
  \usepackage{textcomp} % provide euro and other symbols
\else % if luatex or xetex
  \usepackage{unicode-math}
  \defaultfontfeatures{Scale=MatchLowercase}
  \defaultfontfeatures[\rmfamily]{Ligatures=TeX,Scale=1}
\fi
\usepackage{lmodern}
\ifPDFTeX\else  
    % xetex/luatex font selection
\fi
% Use upquote if available, for straight quotes in verbatim environments
\IfFileExists{upquote.sty}{\usepackage{upquote}}{}
\IfFileExists{microtype.sty}{% use microtype if available
  \usepackage[]{microtype}
  \UseMicrotypeSet[protrusion]{basicmath} % disable protrusion for tt fonts
}{}
\makeatletter
\@ifundefined{KOMAClassName}{% if non-KOMA class
  \IfFileExists{parskip.sty}{%
    \usepackage{parskip}
  }{% else
    \setlength{\parindent}{0pt}
    \setlength{\parskip}{6pt plus 2pt minus 1pt}}
}{% if KOMA class
  \KOMAoptions{parskip=half}}
\makeatother
\usepackage{xcolor}
\setlength{\emergencystretch}{3em} % prevent overfull lines
\setcounter{secnumdepth}{5}
% Make \paragraph and \subparagraph free-standing
\makeatletter
\ifx\paragraph\undefined\else
  \let\oldparagraph\paragraph
  \renewcommand{\paragraph}{
    \@ifstar
      \xxxParagraphStar
      \xxxParagraphNoStar
  }
  \newcommand{\xxxParagraphStar}[1]{\oldparagraph*{#1}\mbox{}}
  \newcommand{\xxxParagraphNoStar}[1]{\oldparagraph{#1}\mbox{}}
\fi
\ifx\subparagraph\undefined\else
  \let\oldsubparagraph\subparagraph
  \renewcommand{\subparagraph}{
    \@ifstar
      \xxxSubParagraphStar
      \xxxSubParagraphNoStar
  }
  \newcommand{\xxxSubParagraphStar}[1]{\oldsubparagraph*{#1}\mbox{}}
  \newcommand{\xxxSubParagraphNoStar}[1]{\oldsubparagraph{#1}\mbox{}}
\fi
\makeatother

\usepackage{color}
\usepackage{fancyvrb}
\newcommand{\VerbBar}{|}
\newcommand{\VERB}{\Verb[commandchars=\\\{\}]}
\DefineVerbatimEnvironment{Highlighting}{Verbatim}{commandchars=\\\{\}}
% Add ',fontsize=\small' for more characters per line
\usepackage{framed}
\definecolor{shadecolor}{RGB}{241,243,245}
\newenvironment{Shaded}{\begin{snugshade}}{\end{snugshade}}
\newcommand{\AlertTok}[1]{\textcolor[rgb]{0.68,0.00,0.00}{#1}}
\newcommand{\AnnotationTok}[1]{\textcolor[rgb]{0.37,0.37,0.37}{#1}}
\newcommand{\AttributeTok}[1]{\textcolor[rgb]{0.40,0.45,0.13}{#1}}
\newcommand{\BaseNTok}[1]{\textcolor[rgb]{0.68,0.00,0.00}{#1}}
\newcommand{\BuiltInTok}[1]{\textcolor[rgb]{0.00,0.23,0.31}{#1}}
\newcommand{\CharTok}[1]{\textcolor[rgb]{0.13,0.47,0.30}{#1}}
\newcommand{\CommentTok}[1]{\textcolor[rgb]{0.37,0.37,0.37}{#1}}
\newcommand{\CommentVarTok}[1]{\textcolor[rgb]{0.37,0.37,0.37}{\textit{#1}}}
\newcommand{\ConstantTok}[1]{\textcolor[rgb]{0.56,0.35,0.01}{#1}}
\newcommand{\ControlFlowTok}[1]{\textcolor[rgb]{0.00,0.23,0.31}{\textbf{#1}}}
\newcommand{\DataTypeTok}[1]{\textcolor[rgb]{0.68,0.00,0.00}{#1}}
\newcommand{\DecValTok}[1]{\textcolor[rgb]{0.68,0.00,0.00}{#1}}
\newcommand{\DocumentationTok}[1]{\textcolor[rgb]{0.37,0.37,0.37}{\textit{#1}}}
\newcommand{\ErrorTok}[1]{\textcolor[rgb]{0.68,0.00,0.00}{#1}}
\newcommand{\ExtensionTok}[1]{\textcolor[rgb]{0.00,0.23,0.31}{#1}}
\newcommand{\FloatTok}[1]{\textcolor[rgb]{0.68,0.00,0.00}{#1}}
\newcommand{\FunctionTok}[1]{\textcolor[rgb]{0.28,0.35,0.67}{#1}}
\newcommand{\ImportTok}[1]{\textcolor[rgb]{0.00,0.46,0.62}{#1}}
\newcommand{\InformationTok}[1]{\textcolor[rgb]{0.37,0.37,0.37}{#1}}
\newcommand{\KeywordTok}[1]{\textcolor[rgb]{0.00,0.23,0.31}{\textbf{#1}}}
\newcommand{\NormalTok}[1]{\textcolor[rgb]{0.00,0.23,0.31}{#1}}
\newcommand{\OperatorTok}[1]{\textcolor[rgb]{0.37,0.37,0.37}{#1}}
\newcommand{\OtherTok}[1]{\textcolor[rgb]{0.00,0.23,0.31}{#1}}
\newcommand{\PreprocessorTok}[1]{\textcolor[rgb]{0.68,0.00,0.00}{#1}}
\newcommand{\RegionMarkerTok}[1]{\textcolor[rgb]{0.00,0.23,0.31}{#1}}
\newcommand{\SpecialCharTok}[1]{\textcolor[rgb]{0.37,0.37,0.37}{#1}}
\newcommand{\SpecialStringTok}[1]{\textcolor[rgb]{0.13,0.47,0.30}{#1}}
\newcommand{\StringTok}[1]{\textcolor[rgb]{0.13,0.47,0.30}{#1}}
\newcommand{\VariableTok}[1]{\textcolor[rgb]{0.07,0.07,0.07}{#1}}
\newcommand{\VerbatimStringTok}[1]{\textcolor[rgb]{0.13,0.47,0.30}{#1}}
\newcommand{\WarningTok}[1]{\textcolor[rgb]{0.37,0.37,0.37}{\textit{#1}}}

\providecommand{\tightlist}{%
  \setlength{\itemsep}{0pt}\setlength{\parskip}{0pt}}\usepackage{longtable,booktabs,array}
\usepackage{calc} % for calculating minipage widths
% Correct order of tables after \paragraph or \subparagraph
\usepackage{etoolbox}
\makeatletter
\patchcmd\longtable{\par}{\if@noskipsec\mbox{}\fi\par}{}{}
\makeatother
% Allow footnotes in longtable head/foot
\IfFileExists{footnotehyper.sty}{\usepackage{footnotehyper}}{\usepackage{footnote}}
\makesavenoteenv{longtable}
\usepackage{graphicx}
\makeatletter
\def\maxwidth{\ifdim\Gin@nat@width>\linewidth\linewidth\else\Gin@nat@width\fi}
\def\maxheight{\ifdim\Gin@nat@height>\textheight\textheight\else\Gin@nat@height\fi}
\makeatother
% Scale images if necessary, so that they will not overflow the page
% margins by default, and it is still possible to overwrite the defaults
% using explicit options in \includegraphics[width, height, ...]{}
\setkeys{Gin}{width=\maxwidth,height=\maxheight,keepaspectratio}
% Set default figure placement to htbp
\makeatletter
\def\fps@figure{htbp}
\makeatother

\usepackage{booktabs}
\usepackage{longtable}
\usepackage{array}
\usepackage{multirow}
\usepackage{wrapfig}
\usepackage{float}
\usepackage{colortbl}
\usepackage{pdflscape}
\usepackage{tabu}
\usepackage{threeparttable}
\usepackage{threeparttablex}
\usepackage[normalem]{ulem}
\usepackage{makecell}
\usepackage{xcolor}
\usepackage{siunitx}

    \newcolumntype{d}{S[
      table-align-text-before=false,
      table-align-text-after=false,
      input-symbols={-,\*+()}
    ]}
  
\usepackage[a4paper,top=7cm,bottom=5cm,left=3cm,right=3cm]{geometry}  % Load geometry package with custom margins
\usepackage{afterpage}
\usepackage{tabularray}
\usepackage{rotating}  % Rotating tables (use this instead of tblrlibrotating.sty)
\usepackage{lipsum}
\usepackage{booktabs}
\usepackage{longtable}
\usepackage{pdflscape}  % Alternative for rotating tables
\usepackage{array}
\usepackage{float}
\usepackage{tabularx}  % Add tabularx for dynamic column width
\floatplacement{table}{H}
\usepackage{setspace}  % For line spacing
\usepackage{caption}   % For caption formatting
\usepackage{siunitx}
\sisetup{group-separator = {,},group-digits = integer}
\usepackage{booktabs}
\usepackage{graphicx}
\usepackage{float}
\usepackage{longtable}
\usepackage{siunitx}
\usepackage{lipsum}
\usepackage{amsmath}
\usepackage{adjustbox}
\makeatletter
\@ifpackageloaded{caption}{}{\usepackage{caption}}
\AtBeginDocument{%
\ifdefined\contentsname
  \renewcommand*\contentsname{Table of contents}
\else
  \newcommand\contentsname{Table of contents}
\fi
\ifdefined\listfigurename
  \renewcommand*\listfigurename{List of Figures}
\else
  \newcommand\listfigurename{List of Figures}
\fi
\ifdefined\listtablename
  \renewcommand*\listtablename{List of Tables}
\else
  \newcommand\listtablename{List of Tables}
\fi
\ifdefined\figurename
  \renewcommand*\figurename{Figure}
\else
  \newcommand\figurename{Figure}
\fi
\ifdefined\tablename
  \renewcommand*\tablename{Table}
\else
  \newcommand\tablename{Table}
\fi
}
\@ifpackageloaded{float}{}{\usepackage{float}}
\floatstyle{ruled}
\@ifundefined{c@chapter}{\newfloat{codelisting}{h}{lop}}{\newfloat{codelisting}{h}{lop}[chapter]}
\floatname{codelisting}{Listing}
\newcommand*\listoflistings{\listof{codelisting}{List of Listings}}
\makeatother
\makeatletter
\makeatother
\makeatletter
\@ifpackageloaded{caption}{}{\usepackage{caption}}
\@ifpackageloaded{subcaption}{}{\usepackage{subcaption}}
\makeatother
\journal{Journal of Business Research}

\ifLuaTeX
  \usepackage{selnolig}  % disable illegal ligatures
\fi
\usepackage[]{natbib}
\bibliographystyle{elsarticle-harv}
\usepackage{bookmark}

\IfFileExists{xurl.sty}{\usepackage{xurl}}{} % add URL line breaks if available
\urlstyle{same} % disable monospaced font for URLs
\hypersetup{
  pdftitle={Template for a Business Research Article},
  pdfauthor={Eric Weisbrod; Bob Security; Cat Memes; Derek Zoolander},
  pdfkeywords={keyword1, keyword2},
  colorlinks=true,
  linkcolor={blue},
  filecolor={Maroon},
  citecolor={Blue},
  urlcolor={Blue},
  pdfcreator={LaTeX via pandoc}}


\setlength{\parindent}{6pt}
\begin{document}

\begin{frontmatter}
\title{Template for a Business Research Article \\\large{Using the
Elsevier Quarto Template} }
\author[1]{Eric Weisbrod%
\corref{cor1}%
}
 \ead{eric.weisbrod@ku.edu} 
\author[2]{Bob Security%
%
}
 \ead{bob@example.com} 
\author[2]{Cat Memes%
%
}
 \ead{cat@example.com} 
\author[1]{Derek Zoolander%
%
}
 \ead{derek@example.com} 

\affiliation[1]{organization={University of Kansas, School of
Business},,postcodesep={}}
\affiliation[2]{organization={Another University, Department
Name},,postcodesep={}}

\cortext[cor1]{Corresponding author}




        
\begin{abstract}
This is the abstract. Lorem ipsum dolor sit amet, consectetur adipiscing
elit. Vestibulum augue turpis, dictum non malesuada a, volutpat eget
velit. Nam placerat turpis purus, eu tristique ex tincidunt et. Mauris
sed augue eget turpis ultrices tincidunt. Sed et mi in leo porta
egestas. Aliquam non laoreet velit. Nunc quis ex vitae eros aliquet
auctor nec ac libero. Duis laoreet sapien eu mi luctus, in bibendum leo
molestie. Sed hendrerit diam diam, ac dapibus nisl volutpat vitae.
Aliquam bibendum varius libero, eu efficitur justo rutrum at. Sed at
tempus elit.
\end{abstract}





\begin{keyword}
    keyword1 \sep 
    keyword2
\end{keyword}
\end{frontmatter}
    
\newgeometry{top=3cm,bottom=3cm,left=3cm,right=3cm}  % Custom margins for the first page


\newpage

\section{Introduction}\label{introduction}

This document is a Quarto template for creating an academic paper
formatted for potential publication in Elsevier journals. If you have
received this document as a PDF, a live version including the code used
to create this document can be found at:
\url{https://github.com/eweisbrod/abr}.

Here are some examples of in-text and parenthetical citations. You can
use latex citation styles inside a Quarto PDF. To add citations to the
paper, you must add the bibtex reference information to the
``Bibliography.bib'' file that is part of the Qaurto project. The
easiest way to do this is to look up the paper you wish to cite on
Google scholar, then click on the ``cite'' link below the paper, select
bibtex, and cut and paste the reference info into the bib file. The
bibtex reference will include a shorthand way to refer to the paper in
the latex cite commands (e.g., ``blankespoor2019individual'').
\citet{blankespoor2019individual} is an in-text cite. There are also
parenthetical cites
\citep[e.g,][]{doyle2006extreme, livnat2006comparing,black2017non, Bradshaw2018}.\footnote{There are also footnotes. }

Here is a new sentence with a parenthetical cite at the end
\citep{easton2024forecasting}.

The second time I cite the paper, I think it will use et al
\citep{easton2024forecasting}.

\section{Background and Hypotheses Development}\label{sec:background}

This template includes some formatting for declaring formal hypotheses
or research questions. I think these commands require some of the
definitions that were set in the preamble above. Here is an example
hypothesis related to the data example that will be used in the tables.

\textbf{Hypothesis 1 (H1) \textbackslash label\{hypo\} :} \emph{Ceteris
paribus, earnings are less persistent for loss firms than profit firms.}

This hypothesis is easy to motivate based on persistent losses driving a
firm out of business, curtailments \citep{lawrence2018losses}, the
abandonment option \citep{hayn1995information}, etc. However, if we want
to define a more open-ended ``research question'' rather than a
``hypothesis,'' we could format it this way:

\textbf{Research Question 1 (RQ1):} \emph{Are losses less persistent
than profits?}

Next, I will provide examples for defining sub-sections and
sub-sub-sections.

\subsection{Example Sub-Section}\label{sec:example}

\subsection{Another sub-section}\label{sec:another}

\subsubsection{This one has a sub-sub-section}\label{sec:another-sub}

\section{Data and Methodology}\label{section-method}

\subsection{Sample Selection}\label{sample-selection}

I downloaded some data from WRDS.

\subsection{Methodology}\label{methodology}

Papers usually have equations. Here is an example DiD equation:

\begin{equation}\phantomsection\label{eq-did}{
\begin{aligned}
    \ln(\text{Dependent Measure}) = {} & \alpha + \beta_1 \text{Post} + \beta_2 \text{Treatment} + \beta_3 (\text{Post} \times \text{Treatment}) \\
    & + A \times \text{Controls} + B \times \text{FE} + \epsilon,
\end{aligned}
}\end{equation}

where \emph{Post} equals 1 for observations in the post-shock period and
0 otherwise, \emph{Treatment} equals 1 for observations with the
treatment and 0 otherwise, \emph{Controls} is a vector of variables
listed as ``Control Variables'' in Appendix {[}
Table~\ref{tbl-variable-def}{]}, and \(FE\) are fixed effects.

Here is the regression equation that we use to test \citep{hypo}, which
was defined in \textbf{?@sec-background}:\\
Footnote: Note that since we have dynamically defined and referred to
{[}H@hyp-losses{]}, you can click on it to jump to the place in the text
where H1 is defined. We can do the same thing for
{[}RQ@researchq-losses{]}.

\begin{equation}\phantomsection\label{eq-losses}{
\begin{aligned}
    ROA_{i,t+1} = {} & \alpha + \beta_1 ROA_{i,t} + \beta_2 LOSS_{i,t} + \beta_3 (ROA_{i,t} \times LOSS_{i,t}) \\
    & + A \times \text{Controls}_{i,t} + B \times \text{FE} + \epsilon_{i,t+1},
\end{aligned}
}\end{equation}

where \(ROA_{i,t+1}\) (\(ROA_{i,t}\)) is return on assets for firm \(i\)
in year \(t+1\) (\(t\)), calculated as earnings before special items
divided by ending total assets. \(LOSS_{i,t}\) is an indicator variable
that equals 1 for observations with negative earnings before special
items and 0 otherwise, and \emph{Controls} is a vector of variables
listed as ``Control Variables'' in \citep[Appendix][]{vars}. \(FE\) are
various fixed effects.

\subsection{Results}\label{results}

We can use LaTeX references to refer/link readers to the tables as we
discuss them. If you click the below table numbers they should take you
to the associated table. These dynamic references will automatically
renumber themselves if additional tables are added or the tables are
reordered. Academic papers rarely use bulleted lists, but here is one
for fun, and to clearly list the tables that are included in this
template:

\begin{itemize}
\item
  Table~\ref{tbl-summary} is an example sample selection table.
\item
  Table~\ref{tbl-frequency} is a basic frequency table.
\item
  Table~\ref{tbl-descriptive} provides descriptive statistics, created
  using R.
\item
  Table~\ref{tbl-corr} provides a correlation matrix, created in R.
\item
  Table~\ref{tbl-regression1} provides an example regression table,
  created in R.

  Here are some examples of inline table references, including
  references to the relevant equation and hypothesis.
  Table~\ref{tbl-corr} and Table~\ref{tbl-regression1} present the
  results from estimating Eq. (\ref{eq:losses}). The significantly
  negative coefficients on \$ LOSS \times ROA\_\{t\}\$ in Columns (4)
  and (5) provide some evidence consistent with H\ref{hyp:losses}.
  However, it seems to be important to control for firm characteristics
  when testing this hypothesis.

  We can also link a graph by \hyperref[fig-FF]{here}.
\end{itemize}

\section{Appendix}\label{appendix}

\begin{verbatim}
\end{verbatim}

\begin{verbatim}
\end{verbatim}

\begin{table}

\caption{\label{tbl-variable-def}Variable Definition}

\centering{

\centering\begingroup\fontsize{10}{12}\selectfont

\resizebox{\ifdim\width>\linewidth\linewidth\else\width\fi}{!}{
\begin{tabular}{>{\raggedright\arraybackslash}p{5cm}>{\raggedright\arraybackslash}p{12cm}}
\toprule
Variable & Definition\\
\midrule
\textbf{Main Dependent and Independent Variables} & \textbf{}\\
ROA\_\{i,t\} & Return on assets for firm \$i\$ in year \$t\$, calculated as earnings before special items divided by ending total assets. In terms of Compustat data items, it is defined as (ib - spi)/at.\\
LOSS\_\{i,t\} & An indicator variable that equals 1 for observations with negative earnings before special items and 0 otherwise.\\
\textbf{Control Variables} & \textbf{}\\
SIZE & Market value of equity as of the end of fiscal year \$t\$ (Source: Compustat).\\
\addlinespace
R\&D & Research and development expense scaled by ending total assets (xrd/at).\\
TA & Ending total assets (at).\\
\bottomrule
\end{tabular}}
\endgroup{}

}

\end{table}%

\begin{table}

\caption{\label{tbl-summary}Sample Selection}

\centering{

\centering
\resizebox{\ifdim\width>\linewidth\linewidth\else\width\fi}{!}{
\begin{tabular}{>{\raggedright\arraybackslash}p{10cm}r}
\toprule
\multicolumn{1}{c}{ } & \multicolumn{1}{c}{THESE ARE NOT REAL NUMBERS} \\
\cmidrule(l{3pt}r{3pt}){2-2}
Description & Observations\\
\midrule
\cellcolor{gray!10}{First I downloaded some data from WRDS} & \cellcolor{gray!10}{194728}\\
After removing financial and utility firms & 136393\\
\cellcolor{gray!10}{After requiring lead ROA data} & \cellcolor{gray!10}{127867}\\
After some other requirements & 86702\\
\cellcolor{gray!10}{With data from another database available} & \cellcolor{gray!10}{71408}\\
\addlinespace
\textbf{Less: firm-years missing data to compute control variables defined in Appendix} & \textbf{-3500}\\
\em{\textbf{\cellcolor{gray!10}{Full Sample}}} & \em{\textbf{\cellcolor{gray!10}{163269}}}\\
\bottomrule
\multicolumn{2}{l}{\textsuperscript{} This table describes the initial sample selection procedure used to collect the data analyzed in our study. Data}\\
\multicolumn{2}{l}{requirements specific to individual tables or analyses are provided in the descriptions of those analyses.}\\
\end{tabular}}

}

\end{table}%

\begin{verbatim}
\end{verbatim}

\begin{longtable}[t]{lrrr}

\caption{\label{tbl-frequency}Summary of Firms by Decade}

\tabularnewline

\toprule
Year & Total Firms & Loss Firms & Pct. Losses\\
\midrule
\cellcolor{gray!10}{1970 - 1979} & \cellcolor{gray!10}{25,101} & \cellcolor{gray!10}{1,978} & \cellcolor{gray!10}{7.88\%}\\
1980 - 1989 & 29,125 & 5,315 & 18.25\%\\
\cellcolor{gray!10}{1990 - 1999} & \cellcolor{gray!10}{38,024} & \cellcolor{gray!10}{9,579} & \cellcolor{gray!10}{25.19\%}\\
2000 - 2009 & 37,100 & 12,822 & 34.56\%\\
\cellcolor{gray!10}{2010 - 2019} & \cellcolor{gray!10}{28,434} & \cellcolor{gray!10}{9,671} & \cellcolor{gray!10}{34.01\%}\\
\addlinespace
2020 - 2022 & 5,514 & 2,516 & 45.63\%\\
\cellcolor{gray!10}{Total} & \cellcolor{gray!10}{163,298} & \cellcolor{gray!10}{41,881} & \cellcolor{gray!10}{25.65\%}\\
\bottomrule

\end{longtable}

\begin{longtable}[t]{lrrrrrrrr}

\caption{\label{tbl-descriptive}Summary Statistics}

\tabularnewline

\toprule
 & N & Mean & SD & Min & P25 & Median & P75 & Max\\
\midrule
roa\_lead\_1 & 163,298 & -0.002 & 0.173 & -0.898 & -0.009 & 0.040 & 0.078 & 0.237\\
roa & 163,298 & 0.006 & 0.159 & -0.803 & -0.003 & 0.042 & 0.079 & 0.243\\
loss & 163,298 & 0.256 & 0.437 & 0.000 & 0.000 & 0.000 & 1.000 & 1.000\\
rd & 163,298 & 0.040 & 0.086 & 0.000 & 0.000 & 0.000 & 0.038 & 0.511\\
at & 163,298 & 1449.837 & 4555.996 & 11.021 & 41.910 & 142.534 & 679.370 & 33637.256\\
\addlinespace
mve & 163,298 & 1568.688 & 5317.423 & 2.127 & 29.790 & 125.510 & 651.281 & 40596.033\\
\bottomrule

\end{longtable}

\begin{verbatim}
\end{verbatim}

\begin{table}

\caption{\label{tbl-corr}Correlation Matrix}

\centering{

\centering
\resizebox{\ifdim\width>\linewidth\linewidth\else\width\fi}{!}{
\begin{tabular}[t]{lrrrrrr}
\toprule
  & roa\_lead\_1 & roa & loss & rd & at & mve\\
\midrule
\cellcolor{gray!10}{roa\_lead\_1} & \cellcolor{gray!10}{1.000} & \cellcolor{gray!10}{0.771} & \cellcolor{gray!10}{-0.569} & \cellcolor{gray!10}{-0.506} & \cellcolor{gray!10}{0.096} & \cellcolor{gray!10}{0.118}\\
roa & 0.771 & 1.000 & -0.688 & -0.570 & 0.094 & 0.119\\
\cellcolor{gray!10}{loss} & \cellcolor{gray!10}{-0.594} & \cellcolor{gray!10}{-0.756} & \cellcolor{gray!10}{1.000} & \cellcolor{gray!10}{0.354} & \cellcolor{gray!10}{-0.106} & \cellcolor{gray!10}{-0.114}\\
rd & -0.105 & -0.111 & 0.202 & 1.000 & -0.068 & -0.017\\
\cellcolor{gray!10}{at} & \cellcolor{gray!10}{0.159} & \cellcolor{gray!10}{0.153} & \cellcolor{gray!10}{-0.201} & \cellcolor{gray!10}{-0.067} & \cellcolor{gray!10}{1.000} & \cellcolor{gray!10}{0.854}\\
\addlinespace
mve & 0.252 & 0.268 & -0.197 & 0.122 & 0.852 & 1.000\\
\bottomrule
\end{tabular}}

}

\end{table}%

\begin{Shaded}
\begin{Highlighting}[]

\end{Highlighting}
\end{Shaded}

\begin{table}

\caption{\label{tbl-regression1}Regression}

\centering{

\centering
\begin{tabular}[t]{lccccc}
\toprule
  & Base & No FE & Year FE & Two-Way FE & With Controls\\
\midrule
ROA[t] & \num{0.839}*** & \num{0.756}*** & \num{0.769}*** & \num{0.639}*** & \num{0.624}***\\
 & (\num{0.002}) & (\num{0.006}) & (\num{0.015}) & (\num{0.017}) & (\num{0.018})\\
LOSS &  & \num{-0.030}*** & \num{-0.028}*** & \num{-0.015}*** & \num{-0.017}***\\
 &  & (\num{0.001}) & (\num{0.004}) & (\num{0.002}) & (\num{0.002})\\
ROA[t] × LOSS &  & \num{0.032}*** & \num{0.012} & \num{-0.285}*** & \num{-0.294}***\\
 &  & (\num{0.007}) & (\num{0.022}) & (\num{0.021}) & (\num{0.023})\\
\midrule
Num.Obs. & \num{163298} & \num{163298} & \num{163298} & \num{161635} & \num{161635}\\
R2 & \num{0.594} & \num{0.597} & \num{0.603} & \num{0.707} & \num{0.707}\\
R2 Within &  &  & \num{0.580} & \num{0.184} & \num{0.186}\\
\bottomrule
\multicolumn{6}{l}{\rule{0pt}{1em}* p $<$ 0.1, ** p $<$ 0.05, *** p $<$ 0.01}\\
\multicolumn{6}{l}{\rule{0pt}{1em}Standard errors in parentheses}\\
\multicolumn{6}{l}{\rule{0pt}{1em}* p<0.1, ** p<0.05, *** p<0.01}\\
\end{tabular}

}

\end{table}%

\begin{verbatim}
\end{verbatim}

\begin{figure}

\caption{\label{fig-FF}\textbf{Frequency of Losses by Industry}}

\centering{

\includegraphics{quarto_final_khairul_files/figure-pdf/fig-FF-1.pdf}

}

\end{figure}%

\newpage

\begin{figure}

\caption{\label{fig-2}Comparison of Loss Frequency and ROA Persistence}

\begin{minipage}{0.50\linewidth}

\centering{

\includegraphics{quarto_final_khairul_files/figure-pdf/fig-2-1.pdf}

}

\subcaption{\label{fig-2-1}Frequency of Losses by Size Quintile}

\end{minipage}%
%
\begin{minipage}{0.50\linewidth}

\centering{

\includegraphics{quarto_final_khairul_files/figure-pdf/fig-2-2.pdf}

}

\subcaption{\label{fig-2-2}ROA Persistence by Loss Status}

\end{minipage}%

\end{figure}%


  \bibliography{bibliography.bib}


\afterpage{\restoregeometry}  % Restore default margins after the first page


\end{document}
